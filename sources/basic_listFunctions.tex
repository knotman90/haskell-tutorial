\section{Basics - List Functions}
\frame{\sectionpage}

\subsection{List Functions - length,++ }
	\begin{frame}[fragile]\frametitle{Lists}
		List is the most used Data structure in Haskell
		\begin{itemize}
		\item Homogenous - Only objects of the same type
		\item Denoted by $[$ CONTENT OF THE LIST $]$
		\item $[$ $[$"passions"$]$, $[$"poetry"$]$, $[$"and"$]$, $[$"the"$]$,
		$[$"ego"$]$ $[$"have"$]$, $[$"been"$]$, $[$"seen"$]$, $[$"as"$]$, $[$"perpetual"$]$
		$[$"explosions?$]$$]$
		\item String are \textbf{List of Char}. We can use list function of
		strings
		\end{itemize}
		\begin{exampleblock}{lenght}
		\texttt{length} is a function that return the length of a List
		 \begin{lstlisting}[language=Haskell,basicstyle=\footnotesize\ttfamily]
			length [1,2,3,4]
			length "Hi guys" 
		\end{lstlisting}
		\end{exampleblock}
	\end{frame}
	
	\begin{frame}[fragile]\frametitle{Let's try them}
		\begin{itemize}
		  \item head, last, init, tail
		  \item map
		  \item \textbf{fold}s are very important but need separate tutorial!
		  \item find
		  replicate, cycle, take(while), drop(while) 
		  \item maximum 
			\end{itemize}
			
		
	\end{frame}
	
		
	\begin{frame}[fragile]\frametitle{Concat}
		A common task is to merge two list. Done using the \textbf{++} operator
		\begin{itemize}
		  \item 
		  \begin{lstlisting}[language=Haskell,basicstyle=\footnotesize\ttfamily]
			[1..3] ++ [4..10], "Hi" ++ "Guys"
		\end{lstlisting}
		\item When possible use (:) instead of (++), the list concatenation operator.
		It's much more faster!
		\end{itemize}
		
	\end{frame}
	