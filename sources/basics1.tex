	\section{Basics - Syntax}
	\frame{\sectionpage}
	
	\subsection{Arithmetic And Boolean algebra}
	
	\begin{frame}[fragile]\frametitle{Syntax Basics}
	\begin{itemize}
	\item Arithmetic and Boolean algebra works as expected
	\begin{lstlisting}[language=Haskell,basicstyle=\footnotesize\ttfamily]
			v1 = 12
			v2 = mod (v1+3) 10 
			v3 = not $ True || (v2>=v1) --not (True || (v2>=v1))
	\end{lstlisting}
	\pause
	\item Function definition is made up of two part: type and body. The body is
	made up of several \emph{clause} that are evaluated (pattern matched)
	\textbf{top to bottom}.
	\begin{lstlisting}[language=Haskell,basicstyle=\footnotesize\ttfamily,
	numbers=left] exp :: Integer -> Integer -> Integer
			exp _ 0 = 1
			exp 0 _ = 0
			exp a b = a * (exp a (b-1)) 
	\end{lstlisting}
	\textbf{What if we swap line 2 and 3?}
	\pause		
	\item Comments:
	 \begin{lstlisting}[language=Haskell,basicstyle=\footnotesize\ttfamily]
		--this is an inline comment
		{-
			All in here is comment
		-}				
	\end{lstlisting}
	\end{itemize}
	\end{frame}
	
	
	\subsection{Guards, where, let}
	\begin{frame}[fragile]\frametitle{Guards, where, let}
		\begin{itemize}
			\item \texttt{Guards},\texttt{let} and \texttt{where} constructs
			\begin{lstlisting}[language=Haskell,basicstyle=\footnotesize\ttfamily,numbers=left]
			fastExp :: Integer -> Integer-> Integer
					fastExp _ 0 = 1
					fastExp a 1 = a
					fastExp a b 
					 	|b < 0 = undefined
					 	|even b = res*res
					 	|otherwise = let next=(fastExp a (b-1)) in (a * next)  
						  	where res=(fastExp a (div b 2))
			\end{lstlisting}
			Suppose we execute \emph{fastExp 2 7}. The call stack would be
			\begin{itemize}
			  \item  \texttt{fastExp 2 7} line 7 pattern match
			  \pause
			  \item  \texttt{fastExp 2 6} line 6 pattern match
			  \pause
			  \item  \texttt{fastExp 2 3} line 7 pattern match
			 \pause
			  \item  \texttt{fastExp 2 2} line 6 pattern match
			 \pause
			  \item  \texttt{fastExp 2 1} line 3 pattern match, STOP RECURSION
			 \end{itemize}
			 \pause
			 In contrast to \texttt{where}, \texttt{let} are \texttt{expressions} and can
			 be used anywhere\footnote{Here for more informations:
			 https://wiki.haskell.org/Let\_vs\_Where}.

	
		\end{itemize}
	\end{frame}
	
	
	\subsection{if and case construct}
		\begin{frame}[fragile]\frametitle{\texttt{If}, \texttt{case}}
		\begin{itemize}
			\item \texttt{if} construct \
			works as expected
			\begin{lstlisting}[language=Haskell,basicstyle=\footnotesize\ttfamily,numbers=left]
			div' n d = if d==0 then Nothing else Just (n/d)
			\end{lstlisting}

			\item \texttt{case} construct \\
			 Useful when we don't wish to define a function every time we need to
			 do pattern matching.
			 
			\begin{lstlisting}[language=Haskell,basicstyle=\footnotesize\ttfamily]
			f p11 ... p1k = e1
			...
			f pn1 ... pnk = en
			--where each pij is a pattern, 
			--is semantically equivalent to:
			f x1 x2 ... xk = case (x1, ..., xk) of
			(p11, ..., p1k) -> e1
			...
			(pn1, ..., pnk) -> en
			\end{lstlisting}
			All patterns of a function return the same type hence all the RHS of the
			\texttt{case} have the same type
	
		\end{itemize}
	\end{frame}
	
	\begin{frame}[fragile]\frametitle{\texttt{case} construct: example}
		\begin{exampleblock}{case construct example}
		Pattern match ``outside'' the function definition. Note that all the cases
		return the same type (a list of $b$'s in this case)
		\begin{lstlisting}[language=Haskell,basicstyle=\footnotesize\ttfamily] 
		cE :: (Ord a) :: a -> a -> [b]
		cE a b xs = case (a `compare` b,xs) of
        (_,[]) -> []
        (LT,xs) -> init xs
        (GT,xs) -> tail xs
        (EQ,xs) -> [head xs]
				
			\end{lstlisting}
		\end{exampleblock}
	\end{frame}
	
	\subsection{Ranges}
	\begin{frame}[fragile]\frametitle{Ranges}
		\begin{exampleblock}{ranges}
		Shortcut for  listing stuff that can be enumerated. What if we need to test if
		a string contains a letter up to the lower case$j$? (Explicitly list all the
		letters is not the correct answer).
		\begin{lstlisting}[language=Haskell,basicstyle=\footnotesize\ttfamily] 
			['a'..'j'] -- results in "abcdefghij" (String are [Char])			
		\end{lstlisting}
		It work even in construction infinite list
		\begin{lstlisting}[language=Haskell,basicstyle=\footnotesize\ttfamily] 
			[1,3..] -- results in [1,3,5,7,9,11,13,15......]			
		\end{lstlisting}
		and because of laziness we can (safely) do
		\begin{lstlisting}[language=Haskell,basicstyle=\footnotesize\ttfamily] 
			take 10 [1,3..] 		
		\end{lstlisting}
		\end{exampleblock}
		
	\end{frame}
	
	\subsection{List}
	\begin{frame}[fragile]\frametitle{List are useful!}
	\begin{itemize}
	  \item Colletcion of elements of the \textbf{SAME TYPE}.
	  \item Delimited by square brackets and elements separated by commas.
	  \item List che be \textit{consed}. The \textbf{cons} operator (:) is used to
	  incrementally build list putting an element at its head.
	  \item empty list is $[]$
	  \item cons is a function that takes two parameter 
	  \begin{lstlisting}[language=Haskell,basicstyle=\footnotesize\ttfamily] 
			(:) :: a -> [a] -> [a]
	\end{lstlisting} 
		
	  \begin{lstlisting}[language=Haskell,basicstyle=\footnotesize\ttfamily] 
			1:2:3:4:[]	
	\end{lstlisting} 
	

	\end{itemize}
		
			
		
	\end{frame}
	
	
	\begin{frame}[fragile]\frametitle{List Comprehension}
		\begin{exampleblock}{list comprehension}
		It is a familiar concept for those who already have some experience in python
		It resambles the mathematical set specification. For instance let's compute
		the list of the factorial of the natural numbers 
		 \begin{lstlisting}[language=Haskell,basicstyle=\footnotesize\ttfamily] 
		 [product [2..x] | x<-[1..]] 		
		\end{lstlisting} 
		\end{exampleblock}
		More examples:
		\begin{lstlisting}[language=Haskell,basicstyle=\footnotesize\ttfamily] 
		 [[2..x*2] | x<-[1..]]
		 [filter (even) [2..x] | x<-[1..]]
		  --:m Data.Char (ord)
		 [let p=y*x in if even p then (negate p) else 
		 	(p*2) |x<-[1..10], y<-(map ord ['a'..'z'])]
		 	--:m Data.List (nub) 
		nub $ map (\(x,y,z) -> z) [(a,b,c) | a<-[1..20],b<-[1..20],
		c<-[1..20], a^2+b^2==c^2, a+b+c>10]
		\end{lstlisting}
			
		
	\end{frame}
	
	
	\subsection{Lambda Functions}
	\begin{frame}[fragile]\frametitle{Lambda functions - The Idea}
		\begin{itemize}
		  \item Anonymous functions i.e. no need to give it a name
		  \item $\lambda y x \rightarrow$ $2x+x^y$ translates in
		   \begin{lstlisting}[language=Haskell] 
			(\x y -> 2*x + x^y)
		\end{lstlisting}
		\item Usually used withing high order function context.
		\begin{lstlisting}[language=Haskell,basicstyle=\footnotesize\ttfamily] 
			map (\x -> x*x-3) [1,10..300]
			map (\x -> let p = ord x in if even p then p else p^2) 
			      "Lambda functions are cool!"
		\end{lstlisting}
		\end{itemize}
		
		
		\begin{itemize}
		  \item $f= (\backslash x_1 .. x_n  -> exp(x_1..x_n)) (v_1, ..,v_k) $
		   substitute each occurence of the free variable $x_i$ with the
		  value $v_i$. If $k<n$ f is again a function.
		  \item 
		\begin{lstlisting}[language=Haskell,basicstyle=\footnotesize\ttfamily] 
			let f = (\x y z -> x+y+z)
			let sum3 = f 2 3 = (\z -> 2+3+z) --again a function
			sum23z 4 -> = 9 
		\end{lstlisting}		  
		\end{itemize}
			
		
	\end{frame}
	
	
	