\section{Find Best Variance - Stock Data}
\subsection{I/O - Find Best Variance}
\frame{\sectionpage}

\begin{frame}[fragile]\frametitle{Best Variance Day}

\begin{block}{Problem Statement}
Write a program that read a file containing daily stock data.
Each line of the file records data regarding prices of a good registered at
regular time interval during each day.
Fine the day which have the maximum variance between opnening and closing price
(first and last price record).
\end{block} 
\textbf{File content:}
\begin{lstlisting}[language=Haskell,basicstyle=\footnotesize\ttfamily]
2012-03-30,32.40,32.41,32.04,32.26,31749400,32.26
2012-03-29,32.06,32.19,31.81,32.12,37038500,32.12
2012-03-28,32.52,32.70,32.04,32.19,41344800,32.19
		\end{lstlisting}

\end{frame}


\begin{frame}[fragile]\frametitle{Solution}

\textbf{The Solution.} \textit{cabal install split}
\begin{lstlisting}[language=Haskell,basicstyle=\footnotesize\ttfamily]
module Main where
import System.Environment (getArgs)
import Data.List.Split (splitOn)
import Data.List (maximumBy)

--main entry point
main = do
	(fileName:_) <- getArgs 
	strF <- readFile fileName
	putStrLn $ maxDay strF	

maxDay ::String -> String
maxDay s = snd $ maximum  ss
 where 
  ss = map (var . (splitOn ",")) $ lines s

var xs = abs diff
  where diff=((read (xs!!1)) - (read (last xs)),head xs)
\end{lstlisting}


\end{frame}
