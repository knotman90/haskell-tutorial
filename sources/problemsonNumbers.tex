\section{Poblem on Numbers}
\frame{\sectionpage}
\subsection{Primality Test}
\begin{frame}[fragile]\frametitle{Primality Tes}


\begin{block}{Problem Statement}
			Determine whether a given integer number is prime.
\end{block} \pause	
\textbf{Examples:}
		\begin{lstlisting}[language=Haskell,basicstyle=\footnotesize\ttfamily]
					isPrime  57601 =  True
					isPrime  1235  =  False
		\end{lstlisting}
\pause
\begin{alertblock}{Solution}
			\begin{lstlisting}[language=Haskell,basicstyle=\footnotesize\ttfamily]
					isPrime l k 
						| k > l = False
						| mod l k ==0 = False
						| otherwise = isPrime l (k+1) 
					
		\end{lstlisting}
\end{alertblock}	

\end{frame}

\subsection{Greatest common divisor}
\begin{frame}[fragile]\frametitle{GCD}


\begin{block}{Problem Statement}
			Implement the Euclid Method to find the greatest common divisor of two
			integer.
			 \end{block} \pause	
\textbf{Examples:}
		\begin{lstlisting}[language=Haskell,basicstyle=\footnotesize\ttfamily]
					gcd' 30 12 	  =  6
					gcd' 5 25     =  5
		\end{lstlisting}
\pause
\begin{alertblock}{Solution}
			\begin{lstlisting}[language=Haskell,basicstyle=\footnotesize\ttfamily]
					gcd' 0 y = y
					gcd' x y = gcd' (mod y x) x
		\end{lstlisting}
\end{alertblock}	

\end{frame}

\subsection{Euler's torient}
\begin{frame}[fragile]\frametitle{Totient function}

\begin{block}{Problem Statement}
Calculate Euler's totient function phi(m).

Euler's so-called totient function $\phi(m)$ is defined as the number of
positive integers $r$ $(1 \leq r < m)$ that are \textbf{coprime} to m.
\end{block} \pause
\textbf{Examples:}
\begin{lstlisting}[language=Haskell,basicstyle=\footnotesize\ttfamily]
			totient 10 	  = 4 
			totient 57601 = 57600 --57601 is prime^^
		\end{lstlisting}
\pause
\begin{alertblock}{Solution}
			\begin{lstlisting}[language=Haskell,basicstyle=\footnotesize\ttfamily]
					totient n = length [e | e <- [1..n], coprime e n]
						where coprime e n = gcd n e ==1
		\end{lstlisting}
\end{alertblock}	

\end{frame}
