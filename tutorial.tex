\documentclass[xcolor=table]{beamer}
\setbeamertemplate{navigation symbols}{}
\usepackage{xcolor}
\usepackage[table,xcdraw]{xcolor}
\usepackage{xparse}

\usepackage{listings}
\lstset{basicstyle=\ttfamily,
  showstringspaces=false,
  commentstyle=\color{red},
  keywordstyle=\color{blue}
}

\newbox\FBox
\NewDocumentCommand\Highlight{O{black}O{white}mO{0.5pt}O{0pt}O{0pt}}{%
    \setlength\fboxsep{#4}\sbox\FBox{\fcolorbox{#1}{#2}{#3\rule[-#5]{0pt}{#6}}}\usebox\FBox}


\lstset{language=Haskell,
	basicstyle=\small\ttfamily,
  showstringspaces=false,
  commentstyle=\color{red},
  keywordstyle=\color{blue},
 aboveskip=-0.2,
  belowskip=-0.8
}
\usetheme{default}
\usecolortheme{seahorse}

\beamersetuncovermixins{\opaqueness<1>{25}}{\opaqueness<2->{15}}
\title{A Journey in Functional Programming}
\subtitle{An introduction to Haskell}
\author{Davide Spataro\inst{1}}
\institute[Universities of]
{
\inst{1}%
Department of mathematics And Computer Science\\
Univeristy of Calabria}

\begin{document}
\date{\today} 


\begin{frame}
\titlepage
\end{frame}

\begin{frame}[allowframebreaks]\frametitle{Table of contents}\tableofcontents
\end{frame} 

		
\section{Introduction - Syntax and Types} 
\frame{\sectionpage}



\subsection{Functional Programming}

\begin{frame}\frametitle{Functional Programming} 
		
		\begin{columns}[T] % align columns
			\begin{column}{.72\textwidth}
				\begin{block}{Definition and Intuitive idea}
				\begin{itemize}
					  \item Computation is just \textbf{function evaluation} $\neq$ \textbf{program
					  state manipulation}.
					  \item Based on $\lambda-$calculus that is an alternative (to set theory)
					  and convenient formalization of logic and mathematics for expressing
					  \textbf{computation}
					  \item Logic deduction $\Leftrightarrow$ $\lambda-$calculus thanks to the
					  Curry-Howard correnspondence.
					  \item A program is a proof!
				\end{itemize}
				\end{block}

			\end{column}%
				\hfill%
				\begin{column}{.28\textwidth}
				\begin{figure}
					\centering
					\includegraphics[scale=0.5]{images/Alonzo_Church.jpg}
					\caption{Alonzo-Church, father of $\lambda-$calculus}
				\end{figure}
			\end{column}%
		\end{columns}
	\end{frame}

\begin{frame}[fragile]\frametitle{Imperative vs Functional} 
		\begin{itemize}
		\item Imperative
		  \begin{itemize}
		    \item Focus on low-level \textbf{how}!
		    \item A program is an ordered sequence of instructions
		    \item Modifies/track the program's state
		\end{itemize}
	\end{itemize}
	
	\begin{itemize}
		\item Functional
		  \begin{itemize}
		    \item Focus on High level \textbf{what}!
		    \item Specify high-level transformation/constraint on the desidered result
		    description.
		\end{itemize}
	\end{itemize}
	
	\begin{columns}[T] % align columns
			\begin{column}{.7\textwidth}
				\begin{block}{Imperative, suffer from the so called \textbf{indexitis}}
				\begin{lstlisting}[language=c]
					unsigned int sum=0;
					for(int i=1;i<100;i++)
						 sum+=i;
				\end{lstlisting}
				\end{block}

			\end{column}%
				\hfill%
				\begin{column}{.28\textwidth}
				\begin{block}{Functional}
				\begin{lstlisting}[language=haskell]
					sum [1..99]
				\end{lstlisting}
				\end{block}
			\end{column}%
		\end{columns}
	\end{frame}
	
	
\begin{frame}[fragile]\frametitle{What does this code do?} 
\centering
\begin{lstlisting}[language=c,basicstyle=\footnotesize\ttfamily]
	void function (int *a, int n) {
	    int i, j, p, t;
	    if (n < 2)
	        return;
	    p = a[n / 2];
	    for (i = 0, j = n - 1;; i++, j--) {
	        while (a[i] < p)
	            i++;
	        while (p < a[j])
	            j--;
	        if (i >= j)
	            break;
	        t = a[i];
	        a[i] = a[j];
	        a[j] = t;
	    }
	    function(a, i);
	    function(a + i, n - i);
	}
\end{lstlisting}
\end{frame}	

\begin{frame}[fragile]\frametitle{\ldots and this?} 
\centering
\hline \hfill \\
\begin{lstlisting}[language=haskell,basicstyle=\footnotesize\ttfamily]
function ::(Ord a) =>  [a] -> [a]
function [] = []
function (x:xs) = (function l) ++ [x] ++ (function g)
   where 
	    l = filter (<x) xs
	    g = filter (>=x) xs
\end{lstlisting}
\hline
	\begin{itemize}
	  \item No indices
	  \item No memory/pointer management
	  \item No variable assignment
	\end{itemize}	
\end{frame}	
	
\begin{frame}\frametitle{Imperative vs Functional} 
\begin{table}[h]
\centering
\resizebox{\textwidth}{!}{%
\begin{tabular}{|l|l|l|}
\hline
\textbf{Characteristic}                     & \textbf{Imperative}                        & \textbf{Functional}                                                                            \\ \hline
\textit{\textbf{Programmer focus}}          & Algorithm design                           & What the output look like?                                                                     \\ \hline
\textit{\textbf{State changes}}             & {\color[HTML]{9A0000} Fundamental}         & {\color[HTML]{9A0000} Non-existent}                                                            \\ \hline
\textit{\textbf{Order of execution}}        & Important                                  & \begin{tabular}[c]{@{}l@{}}Low importance\\  (compilers may do much work on this)\end{tabular} \\ \hline
\textit{\textbf{Primary flow control}}      & {\color[HTML]{9A0000} Loops, conditionals} & {\color[HTML]{9A0000} Recursion and Functions}                                                 \\ \hline
\textit{\textbf{Primary data unit}} & structures or classes       
& Functions                                                                                      \\ \hline
\end{tabular}
}
\end{table}

\begin{itemize}
  \item Other pure/quasi-pure languages: Erlang, Scala, F#, LISP.  
\end{itemize}

\end{frame}

\begin{frame}\frametitle{Why Functional Programming? Why Haskell?} 
		\begin{enumerate}
		  \item Haskell's expressive power can improve
		  productivity/understandability/maintanibility
				\begin{itemize}
			    \item Get best from compiled and interpreted languages
			    \item Can understand what complex library does
			  \end{itemize}
		\item Strong typed - Catches bugs at \emph{compile time}
		\item Powerful type inference engine
		\item New Testing metologies
		\item Automatic parallelization due to code purity
		\end{enumerate}
	\end{frame}
	
	\subsection{Tools and Installation}
		\begin{frame}[fragile]\frametitle{Haskell platform}
		A full comprehensive, development environment for
		Haskell\footnote{https://www.haskell.org/platform/index.html}\footnote{http://tryhaskell.org/}.
		
			\begin{block}{Installation}
			\begin{itemize}
			  \item 
				\begin{lstlisting}[language=bash]
					$sudo apt-get install haskell-platform
				\end{lstlisting}
			\end{itemize}
			\end{block}
			
			\begin{description}
			  \item[GHC]
			  	 (Great Glasgow Compiler): State of the art
			  \item[GHCi]
			  	 A read-eval-print loop interpreter
			  \item[Cabal]
			   	Build/distribuite/retrieve libraries
			  \item[Haddock]
			   		A high quality documentation generation tool for Haskell 
			\end{description}
\end{frame}

	\begin{frame}[fragile]\frametitle{What really is Haskell?}
		\begin{block}{Purely Functional language}
			\begin{itemize}
			  \item Functions are first-class object (same things as data)
			\item Deterministic - No Side Effect- same function call $\Rightarrow$ same
			Ouput, EVER!\\
			This \emph{referial transparency} leaves room for compiler optimization and 
			allow to mathematically prove correctness.
			\item safely replace expressions by its (unique) result value
			\item \textbf{Evaluate expression} rather than execute instruction
			\item Function describes what data are, not what what to do to\ldots
			\item Everything (variables, data structures\ldots) is immutable 
			\item Multi-parameters function simply does not exists.
			\end{itemize}
			
		\end{block} 
	
	\end{frame}
	
	\begin{frame}[fragile]\frametitle{Haskell is Lazy}
			\begin{columns}[T] % align columns
			\begin{column}{.67\textwidth}

			\begin{alertblock}{It won't execute anything until is \emph{really
			needed}} 
					\begin{itemize}
					  	\item It is possible to define and work with infinite data structures
					  \item Define new control structure just by defining a function.
					  \item Reasoning about time/space complexity much more complicated
					\end{itemize}
			\end{alertblock}
			

			\end{column}%
			\hfill%
			\begin{column}{.45\textwidth}
			\includegraphics[scale=0.5]{./images/homer}
			\end{column}%
			\end{columns}
			
			
	\end{frame}
	
	
	
\begin{frame}[fragile]\frametitle{Understanding laziness}
	\begin{lstlisting}[language=Haskell]
		lazyEval 0 b = 1
		lazyEval _ b = b
	\end{lstlisting}
	\begin{itemize}
	  \item b never computed if the first parameter is zero
	  \item this call is safe: 
	  \begin{lstlisting}[language=Haskell] 
	  	lazyEval 0	(2^123123123123123123123)
	  \end{lstlisting}
	  \item this is not 
	    \begin{lstlisting}[language=Haskell] 
	    lazyEval 1 	(2^123123123123123123123)
	  \end{lstlisting}
	\end{itemize}
	\begin{exampleblock}{Strict evaluation: parameter are evaluated
	  \textbf{before} to be passed to functions}
	\begin{itemize}
	 
	\end{itemize}
	
	 	\begin{lstlisting}[language=c++,basicstyle=\footnotesize\ttfamily]
	 	 	int cont=0;
	 	 	auto fcall = [] (int a, int b) 
	 	 	{if(a==0) return 1; else return b;};
	 	 	auto f1 = [] () { cont++; return 1};
	 	 	auto f2 = [] () { cont+=10; return 2};
	    	fcall (f1(),f2()));
	  \end{lstlisting} 
	fcall will always  increments $cont$ twice! 
	\end{exampleblock}
			
			
\end{frame}
	
\subsection{Hello world(s)}
	\begin{frame}[fragile]\frametitle{Hello World}
			\begin{block}{Our First Program}
				Create a file \emph{hello.hs} and compile with the followings
				\begin{lstlisting}[language=Haskell]
						main = putStrLn "Hello World with Haskell"
				\end{lstlisting}
				\begin{lstlisting}[language=bash]
						$ghc -o hello hello.hs
				\end{lstlisting}
			\end{block}
			\begin{block}{GHCi}
				Execute and play with GHCi by simply typing
				\begin{lstlisting}[language=Haskell]
						reverse [1..10]
						:t foldl
						[1..]
						(filter (even) .reverse) [1..100]
				\end{lstlisting}
			\end{block}
		\end{frame}
		
		\begin{frame}[fragile]\frametitle{Hello Currying}
		Another example, the $k^{th}$ Fibonacci number (type in GHCi):
		\begin{lstlisting}[language=Haskell,xleftmargin=-1.5em]
			let f a b k = if k==0 then a else f b (a+b) (k-1)
		\end{lstlisting}
		\begin{itemize}
		  \item	Defines a recursive function \emph{f} that takes \emph{a,b,k} as parameters:
		  \item Spaces are important. Are like function call operator
		  \emph{()} in C-like languages.
		  \item Wait, three space in $f\;a\;b\;k$: 3 function
		  calls? YES!. \textbf{Every function in Haskell officially only takes one
		  parameter}. 
		  \item f infact has type
		 \begin{lstlisting}[language=Haskell,xleftmargin=-1.5em]
			f :: Integer->(Integer->(Integer->Integer))
		\end{lstlisting}
		i.e. a function that takes an integer and return (the \emph{-$>$}) a function
		that takes an integer and return \ldots
		 \begin{lstlisting}[language=Haskell,xleftmargin=-1.5em]
			f 0 :: Integer->(Integer->Integer)
			f 0 1 :: Integer->Integer
			f 0 1 10 :: Integer
		\end{lstlisting}
		
		\end{itemize}
		\end{frame}
		
		\begin{frame}[fragile]\frametitle{Hello Currying - 2}
		Currying directly and naturally address the  high-order functions support
		 Haskell machinery.
		 \begin{alertblock}{High-order function:}
		 \begin{itemize}
		   \item Take function as parameter
		   \item returns a function%\footnote{In haskell \textbf{return exp} has a
		   % different meaning respect to the classic: \textit{return execution of the program to
		   % the caller, and report the valfooue of exp}} 
		\end{itemize}
		 \end{alertblock}
		 
		 \begin{exampleblock}{zipwith}
		  \begin{itemize}
		   \item Combines two list of type $a$ and $b$ using a function f that takes a
		   parameter of type a and one of type b and return a value of type $c$,
		   producing a list of elements of type $c$.
		   \item 
		 \begin{lstlisting}[language=Haskell,xleftmargin=-0.3em]
			zipWith :: (a -> b -> c) -> [a] -> [b] -> [c]
		\end{lstlisting}
		\end{itemize}
		 \end{exampleblock}
		
		
		\end{frame}
		
		
		
		\begin{frame}[fragile]\frametitle{Hello Currying - 2}
		
		\begin{lstlisting}[language=Haskell,xleftmargin=-0.3em]
			zipWith :: (a -> b -> c) -> [a] -> [b] -> [c]
			zipWith _ _ [] = []
			zipWith _ [] _ = []
			zipWith f (x:xs) (y:ys) = f x y : zipWith f xs ys
		\end{lstlisting}
		 \begin{exampleblock}{usage examples}
		  \begin{lstlisting}[language=Haskell,xleftmargin=-0.3em]
			zipWith (+) [1,2,3] [4,5,6] = [5,7,9]
			zipWith (*) [1,2,3] [4,5,6] = [4,10,18]
			zipWith (\a b ->(+).(2*))) [1..] [1..]
		\end{lstlisting}
		What about this call? (missing one parameter)
		\begin{lstlisting}[language=Haskell]
		let l = zipWith (*) [1,2,3] 
		l [3,2,1]
		\end{lstlisting}
		 \end{exampleblock}
		
		\end{frame}
		
		
		
		\begin{frame}[fragile]\frametitle{Hello World - 3}
			
			\begin{exampleblock}{Number of distinct powers counting (Project Euler #29)}
			Consider all integer combinations of $a,b$ for $2 \leq a,b
			\leq 100$:
			how many distinct terms are in the sequence generated by $a^b$?
			\hfill \\
			\begin{smashedalign}
			&2^2=4,2^3=8,\Highlight[blue][blue!20]{$2^4=16$},2^5=32 \\
			&3^2=9,3^3=27,3^4=81,3^5=243\\
			&\Highlight[blue][blue!20]{$4^2=16$},4^3=64,4^4=256,4^5=1024\\
			&5^2=25,5^3=125,5^4=625,5^5=3125\\
			\end{smashedalign}
			\end{exampleblock} \pause
			
			\begin{exampleblock}{Na{\"i}ve solution}
			\begin{lstlisting}[language=Haskell]
			np a b = length $ nub l
				where l = [c^d | c<-[2..a],d<-[2..b]]
		\end{lstlisting}
% 				np :: Integer -> Integer -> Int
% 				np a b = let l = [a^b |a<-[2..a],b<-[2..b]] 
% 					 in length (remDup l)
% 	     		where 
% 	     			remDup = (map  head . group . sort)
			\end{exampleblock}
			
		\end{frame}
	
	
		\begin{frame}[fragile]\frametitle{Statically Typed}
		\begin{itemize}
			  \item Haskell is stricly typed
			  \item Helps in thinking and express program structure
			  \item \textbf{Turns run-time errors into compile-time errors}. 		  
			   “If it compiles, it must be correct” is moslty true\footnote{It is still
			   quite possible to have errors in logic even in a type-correct program}. 
		\end{itemize}
		\begin{block}{Abstraction: Every idea, algorithm, and piece of data should
		occur exactly once in your code.}
		Haskell features as parametric polymorphis, typeclasses high-order functions
		greatly aid in fighting repetition.
		
		\end{block}
			
	\end{frame}
	
		\begin{frame}[fragile]\frametitle{What really is Haskell?}
			\begin{exampleblock}{C-like vs Haskell}
			Code as the one that follows
			\begin{lstlisting}[language=C++]
				int acc = 0;
				for ( int i = 0; i < lst.length; i++ )
 					acc	 = acc + 3 * lst[i];
			\end{lstlisting}
			is full of low-level details of iterating over an array by
			keeping track of a current index. It much elegantely translates in:
			\begin{lstlisting}[language=Haskell]
					sum (map (*3) lst)
			\end{lstlisting}
			\end{exampleblock}
			Other examples:
			\begin{lstlisting}[language=Haskell,xleftmargin=-1.5em]
					partition (even) [49, 58, 76, 82, 83, 90]
					--prime number generation
					let pgen (p:xs) = p : pgen [x|x <- xs, x`mod`p > 0]
					take 40 (pgen [2..])
				\end{lstlisting}
	\end{frame}
	

	\section{Basics I}
	\frame{\sectionpage}
	
	\subsection{Syntax}
	
	\begin{frame}[fragile]\frametitle{Syntax Basics}
	\begin{itemize}
	\item Arithmetic and Boolean algebra works as expected
	\begin{lstlisting}[language=Haskell,basicstyle=\footnotesize\ttfamily]
			v1 = 12
			v2 = mod (v1+3) 10 
			v3 = not $ True || (v2>=v1) --not (True || (v2>=v1))
	\end{lstlisting}
	\pause
	\item Function definition is made up of two part: type and body. The body is
	made up of several \emph{clause} that are evaluated (pattern matched)
	\textbf{top to bottom}.
	\begin{lstlisting}[language=Haskell,basicstyle=\footnotesize\ttfamily,
	numbers=left] exp :: Integer -> Integer -> Integer
			exp _ 0 = 1
			exp 0 _ = 0
			exp a b = a * (exp a (b-1)) 
	\end{lstlisting}
	\textbf{What if we swap line 2 and 3?}
	\pause		
	\item Comments:
	 \begin{lstlisting}[language=Haskell,basicstyle=\footnotesize\ttfamily]
		--this is an inline comment
		{-
			All in here is comment
		-}				
	\end{lstlisting}
	\end{itemize}
	\end{frame}
	
	
	
	\begin{frame}[fragile]\frametitle{Guards, where, let}
		\begin{itemize}
			\item \texttt{Guards},\texttt{let} and \texttt{where} constructs
			\begin{lstlisting}[language=Haskell,basicstyle=\footnotesize\ttfamily,numbers=left]
			fastExp :: Integer -> Integer-> Integer
					fastExp _ 0 = 1
					fastExp a 1 = a
					fastExp a b 
					 	|b < 0 = undefined
					 	|even b = res*res
					 	|otherwise = let next=(fastExp a (b-1)) in (a * next)  
						  	where res=(fastExp a (div b 2))
			\end{lstlisting}
			Suppose we execute \emph{fastExp 2 7}. The call stack would be
			\begin{itemize}
			  \item  \texttt{fastExp 2 7} line 7 pattern match
			  \pause
			  \item  \texttt{fastExp 2 6} line 6 pattern match
			  \pause
			  \item  \texttt{fastExp 2 3} line 7 pattern match
			 \pause
			  \item  \texttt{fastExp 2 2} line 6 pattern match
			 \pause
			  \item  \texttt{fastExp 2 1} line 3 pattern match, STOP RECURSION
			 \end{itemize}
			 \pause
			 In contrast to \texttt{where}, \texttt{let} are \texttt{expressions} and can
			 be used anywhere\footnote{Here for more informations:
			 https://wiki.haskell.org/Let\_vs\_Where}.

	
		\end{itemize}
	\end{frame}
	
	
	
		\begin{frame}[fragile]\frametitle{\texttt{If}, \texttt{case}}
		\begin{itemize}
			\item \texttt{if} construct \
			works as expected
			\begin{lstlisting}[language=Haskell,basicstyle=\footnotesize\ttfamily,numbers=left]
			div' n d = if d==0 then Nothing else Just (n/d)
			\end{lstlisting}

			\item \texttt{case} construct \\
			 Useful when we don't wish to define a function every time we need to
			 do pattern matching.
			 
			\begin{lstlisting}[language=Haskell,basicstyle=\footnotesize\ttfamily]
			f p11 ... p1k = e1
			...
			f pn1 ... pnk = en
			--where each pij is a pattern, is semantically equivalent to:
			f x1 x2 ... xk = case (x1, ..., xk) of
			(p11, ..., p1k) -> e1
			...
			(pn1, ..., pnk) -> en
			\end{lstlisting}
			All patterns of a function return the same type hence all the RHS of the
			\texttt{case} have the same type
	
		\end{itemize}
	\end{frame}
	
	\begin{frame}[fragile]\frametitle{\texttt{case} construct: example}
		\begin{exampleblock}{case construct example}
		Pattern match ``outside'' the function definition. Note that all the cases
		return the same type (a list of $b$'s in this case)
		\begin{lstlisting}[language=Haskell,basicstyle=\footnotesize\ttfamily] 
		cE :: (Ord a) :: a -> a -> [b]
		cE a b xs = case (a `compare` b,xs) of
        (_,[]) -> []
        (LT,xs) -> init xs
        (GT,xs) -> tail xs
        (EQ,xs) -> [head xs]
				
			\end{lstlisting}
		\end{exampleblock}
	\end{frame}
	
	\begin{frame}[fragile]\frametitle{Ranges And List Comprehension}
		\begin{exampleblock}{ranges}
		Shortcut for  listing stuff that can be enumerated. What if we need to test if
		a string contains a letter up to the lower case$j$? (Explicitly list all the
		letters is not the correct answer).
		\begin{lstlisting}[language=Haskell,basicstyle=\footnotesize\ttfamily] 
			['a'..'j'] -- results in "abcdefghij" (String are [Char])			
		\end{lstlisting}
		It work even in construction infinite list
		\begin{lstlisting}[language=Haskell,basicstyle=\footnotesize\ttfamily] 
			[1,3..] -- results in [1,3,5,7,9,11,13,15......]			
		\end{lstlisting}
		and because of laziness we can (safely) do
		\begin{lstlisting}[language=Haskell,basicstyle=\footnotesize\ttfamily] 
			take 10 [1,3..] 		
		\end{lstlisting}
		\end{exampleblock}
		\begin{exampleblock}{list comprehension}
		It is a familiar concept for those who already have some experience in python
		It resambles the mathematical set specification. For instance let's compute
		the list of the factorial of the natural numbers 
		 \begin{lstlisting}[language=Haskell,basicstyle=\footnotesize\ttfamily] 
		 [product [2..x] | x<-[1..]] 		
		\end{lstlisting} 
		\end{exampleblock}
		
	\end{frame}
	
	
	
\section{Basics - List Functions}
\frame{\sectionpage}

\subsection{List Functions - length,++ }
	\begin{frame}[fragile]\frametitle{Lists}
		List is the most used Data structure in Haskell
		\begin{itemize}
		\item Homogenous - Only objects of the same type
		\item Denoted by $[$ CONTENT OF THE LIST $]$
		\item $[$ $[$"passions"$]$, $[$"poetry"$]$, $[$"and"$]$, $[$"the"$]$,
		$[$"ego"$]$ $[$"have"$]$, $[$"been"$]$, $[$"seen"$]$, $[$"as"$]$, $[$"perpetual"$]$
		$[$"explosions?$]$$]$
		\item String are \textbf{List of Char}. We can use list function of
		strings
		\end{itemize}
		\begin{exampleblock}{lenght}
		\texttt{length} is a function that return the length of a List
		 \begin{lstlisting}[language=Haskell,basicstyle=\footnotesize\ttfamily]
			length [1,2,3,4]
			length "Hi guys" 
		\end{lstlisting}
		\end{exampleblock}
	\end{frame}

	
	\begin{frame}[fragile]\frametitle{Concat}
		A common task is to merge two list. Done using the \textbf{++} operator
		\begin{itemize}
		  \item 
		  \begin{lstlisting}[language=Haskell,basicstyle=\footnotesize\ttfamily]
			[1..3] ++ [4..10], "Hi" ++ "Guys"
		\end{lstlisting}
		\item When possible use (:) instead of (++), the list concatenation operator.
		It's much more faster!
		\end{itemize}
		
	\end{frame}
\section{Coding - Problems on Lists}
\frame{\sectionpage}

\subsection{Last element}
\begin{frame}[fragile]\frametitle{Last element}


\begin{block}{Problem Statement}
			Given a polymorphic list $l$ of type $[a]$, find the last element of l
			(not using function $last$, I'm sorry).
\end{block}	
\textbf{Examples:}
		\begin{lstlisting}[language=Haskell]
					_last [1,2,3,4] =  4 
		\end{lstlisting} 
		\pause
		\begin{lstlisting}[language=Haskell]
					_last ["programming","haskell","is","cool"]= "cool" 
		\end{lstlisting}
					
\pause \pause
\begin{alertblock}{Solution}
			\begin{lstlisting}[language=Haskell]
					_last :: [a] -> a
					_last [] 	 = error "Undefined operation"
					_last (x:[]) = x
					_last (x:xs) = _last xs 
		\end{lstlisting}
\end{alertblock}	

\end{frame}

\subsection{$k$ìth element}
\begin{frame}[fragile]\frametitle{$k$'th element of a list}


\begin{block}{Problem Statement}
			Find the $k$'th element of a list where the first element has index
			\mathbf{$1$}
\end{block}	
\textbf{Examples:}
		\begin{lstlisting}[language=Haskell,basicstyle=\footnotesize\ttfamily]
					elementAt 2 [3,35,32,33]  =  35 
		\end{lstlisting}
		\pause
		\begin{lstlisting}[language=Haskell,basicstyle=\footnotesize\ttfamily]
					elementAt 3 [('a',97),('b',98),('c',99)] = ('c',99)
					elementAt 4 [('a',97),('b',98),('c',99)] = error "Index out of bound" 
		\end{lstlisting}
					
\pause \pause
\begin{alertblock}{Solution}
			\begin{lstlisting}[language=Haskell]
					elementAt :: Integer -> [a] -> a
					elementAt  _ [] 	 = error "index out of bound"
					elementAt 1 (x:_)	 = x
					elementAt n (_:xs)   = elementAt (n-1) xs 
		\end{lstlisting}
\end{alertblock}	

\end{frame}


\subsection{Palindrome List}
\begin{frame}[fragile]\frametitle{Palindromic List}


\begin{block}{Problem Statement}
			Write a function that returns a boolean value tha indicates whether the input
			list is palindromic or not.
			\mathbf{$1$}
\end{block} \pause	
\textbf{Examples:}
		\begin{lstlisting}[language=Haskell,basicstyle=\footnotesize\ttfamily]
					palindrome  "itopinonavevanonipoti"  =  True
					palindrome  "[1,2,3,3,1]  =  False
		\end{lstlisting}
\pause
\begin{alertblock}{Solution}
			\begin{lstlisting}[language=Haskell,basicstyle=\footnotesize\ttfamily]
					palindrome1 l = l== reverse l
					
					palindrome2 [] = True --empty list is palindrome
					palindrome2 (_:[]) = True --one element is palindrome
					palindrome2 l 
    					| head l /= last l = False
    					| otherwise 	= palindrome2 ((tail . init) l)
					
		\end{lstlisting}
\end{alertblock}	

\end{frame}


\section{Poblem on Numbers}
\frame{\sectionpage}
\subsection{Primality Test}
\begin{frame}[fragile]\frametitle{Primality Tes}


\begin{block}{Problem Statement}
			Determine whether a given integer number is prime.
\end{block} \pause	
\textbf{Examples:}
		\begin{lstlisting}[language=Haskell,basicstyle=\footnotesize\ttfamily]
					isPrime  57601 =  True
					isPrime  1235  =  False
		\end{lstlisting}
\pause
\begin{alertblock}{Solution}
			\begin{lstlisting}[language=Haskell,basicstyle=\footnotesize\ttfamily]
					isPrime l k 
						| k > l = False
						| mod l k ==0 = False
						| otherwise = isPrime l (k+1) 
					
		\end{lstlisting}
\end{alertblock}	

\end{frame}

\subsection{Greatest common divisor}
\begin{frame}[fragile]\frametitle{GCD}


\begin{block}{Problem Statement}
			Implement the Euclid Method to find the greatest common divisor of two
			integer.
			 \end{block} \pause	
\textbf{Examples:}
		\begin{lstlisting}[language=Haskell,basicstyle=\footnotesize\ttfamily]
					gcd' 30 12 	  =  6
					gcd' 5 25     =  5
		\end{lstlisting}
\pause
\begin{alertblock}{Solution}
			\begin{lstlisting}[language=Haskell,basicstyle=\footnotesize\ttfamily]
					gcd' 0 y = y
					gcd' x y = gcd' (mod y x) x
		\end{lstlisting}
\end{alertblock}	

\end{frame}

\subsection{Euler's torient}
\begin{frame}[fragile]\frametitle{Totient function}

\begin{block}{Problem Statement}
Calculate Euler's totient function phi(m).

Euler's so-called totient function $\phi(m)$ is defined as the number of
positive integers $r$ $(1 \leq r < m)$ that are \textbf{coprime} to m.
\end{block} \pause
\textbf{Examples:}
\begin{lstlisting}[language=Haskell,basicstyle=\footnotesize\ttfamily]
			totient 10 	  = 4 
			totient 57601 = 57600 --57601 is prime^^
		\end{lstlisting}
\pause
\begin{alertblock}{Solution}
			\begin{lstlisting}[language=Haskell,basicstyle=\footnotesize\ttfamily]
					totient n = length [e | e <- [1..n], coprime e n]
						where coprime e n = gcd n e ==1
		\end{lstlisting}
\end{alertblock}	

\end{frame}

\section{Find Best Variance - Stock Data}
\subsection{I/O - Find Best Variance}
\frame{\sectionpage}

\begin{frame}[fragile]\frametitle{Best Variance Day}

\begin{block}{Problem Statement}
Write a program that read a file containing daily stock data.
Each line of the file records data regarding prices of a good registered at
regular time interval during each day.
Fine the day which have the maximum variance between opnening and closing price
(first and last price record).
\end{block} 
\textbf{File content:}
\begin{lstlisting}[language=Haskell,basicstyle=\footnotesize\ttfamily]
2012-03-30,32.40,32.41,32.04,32.26,31749400,32.26
2012-03-29,32.06,32.19,31.81,32.12,37038500,32.12
2012-03-28,32.52,32.70,32.04,32.19,41344800,32.19
		\end{lstlisting}

\end{frame}


\begin{frame}[fragile]\frametitle{Solution}

\textbf{The Solution.} \textit{cabal install split}
\begin{lstlisting}[language=Haskell,basicstyle=\footnotesize\ttfamily]
module Main where
import System.Environment (getArgs)
import Data.List.Split (splitOn)
import Data.List (maximumBy)

--main entry point
main = do
	(fileName:_) <- getArgs 
	strF <- readFile fileName
	putStrLn $ maxDay strF	

maxDay ::String -> String
maxDay s = snd $ maximum  ss
 where 
  ss = map (var . (splitOn ",")) $ lines s

var xs = abs diff
  where diff=((read (xs!!1)) - (read (last xs)),head xs)
\end{lstlisting}


\end{frame}

\section{Coding - Project Euler Problem 1 & 2}
\frame{\sectionpage}

\subsection{Problems 1}
	
\begin{frame}[fragile]\frametitle{Problems 1}


\begin{block}{Problem Statement}
			If we list all the natural numbers below 10 that are multiples of
			 3 or 5, we get 3, 5, 6 and 9. The sum of these multiples is 23.
			Find the sum of all the multiples of 3 or 5 below 1000.
\end{block}	
\pause
{\centering
\textbf{How would you solve it using Haskell?}}\pause
	 \begin{lstlisting}[language=Haskell,basicstyle=\footnotesize\ttfamily]]
	problem1' = sum . 
	            filter (\x -> x `mod` 3==0 || x `mod` 5 ==0)
	\end{lstlisting}
\end{frame}
\input{sources/problems26}

\begin{frame}[plain]

  \begin{columns}
    \begin{column}{0.4\textwidth}
      \begin{center}
        \pgfimage[width=\textwidth]{images/haskellIMG.png}          
      \end{center}
    \end{column}
    \begin{column}{0.75\textwidth}
      \begin{center}

        \font\endfont = cmss10 at 15.40mm
        \color{Brown}
        \endfont 
        \baselineskip 20.0mm

        Thank you

      \end{center}    

    \end{column}
  \end{columns}

\end{frame}
	





\end{document}
